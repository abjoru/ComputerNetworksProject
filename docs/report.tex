\documentclass{scrartcl}
\usepackage{listings}
\usepackage{xcolor}
\usepackage{graphicx}
\usepackage{float}
\usepackage[english]{babel}
\usepackage[toc,page]{appendix}
\setlength{\topmargin}{-.5in}
\setlength{\textheight}{9in}
\setlength{\oddsidemargin}{.125in}
\setlength{\textwidth}{6in}
\def\UrlFont{\em}

\colorlet{punct}{red!60!black}
\definecolor{background}{HTML}{EEEEEE}
\definecolor{delim}{RGB}{20,105,176}
\colorlet{numb}{magenta!60!black}

\lstdefinelanguage{json}{
    basicstyle=\small\ttfamily,
    numbers=left,
    numberstyle=\scriptsize,
    stepnumber=1,
    numbersep=8pt,
    showstringspaces=false,
    breaklines=true,
    frame=lines,
    tabsize=2,
    backgroundcolor=\color{background},
    literate=
     *{0}{{{\color{numb}0}}}{1}
      {1}{{{\color{numb}1}}}{1}
      {2}{{{\color{numb}2}}}{1}
      {3}{{{\color{numb}3}}}{1}
      {4}{{{\color{numb}4}}}{1}
      {5}{{{\color{numb}5}}}{1}
      {6}{{{\color{numb}6}}}{1}
      {7}{{{\color{numb}7}}}{1}
      {8}{{{\color{numb}8}}}{1}
      {9}{{{\color{numb}9}}}{1}
      {:}{{{\color{punct}{:}}}}{1}
      {,}{{{\color{punct}{,}}}}{1}
      {\{}{{{\color{delim}{\{}}}}{1}
      {\}}{{{\color{delim}{\}}}}}{1}
      {[}{{{\color{delim}{[}}}}{1}
      {]}{{{\color{delim}{]}}}}{1},
}

\begin{document}
\author{Andreas Bjoru \& Srinivasa Venkatesh\\ Florida Institute of Technology}
\title{CSE 5231 Computer Networks}
\subtitle{Class Project Report, Fall 2013}
\renewcommand{\today}{November 26, 2013}
\maketitle
\tableofcontents
\newpage

\section{Summary}
The objective of this project is to demonstrate the functionality of some of the OSI layers in the communication stack. In particular, key functions performed by layers 2 (Data Link), 3 (Network) and 4 (Transport) is to be emulated for the purpose of transferring a file from one network node to another \cite{assignment}. 

\section{Architecture}
The architecture of the network simulator is fairly simple. The main application class, \texttt{Simulator}, is responsible for reading the topology and creating each network node. A network node, \textit{host or router}, are represented as system threads. Each of these network nodes share a common super-class \texttt{AbstractNetworkNode} that defines common functionality and basic implementation for the OSI layer methods. The OSI layer methods shares the same pattern in which they handle both incoming and outgoing data. This is achieved through the \texttt{Transmit} enum type that is a required argument along with the data payload. \\

The \texttt{physicalLayer(byte[], Transmit, Address)} method will send a given packet to every node attached to the same network, thereby simulating the task of writing a packet on the 'wire'. This is achieved by collecting every node attached to the sender's network address through the use of utility methods on the \texttt{Topology} class. When the physical layer method receives a packet, it will just hand it off to the layer above it.\\

The \texttt{linkLayer(byte[], Transmit, Address)} method will packet a given payload in an \texttt{EthernetFrame} along with the source and destination MAC addresses from the \texttt{Address} argument. The frame will then be handed down to the physical layer method to complete the sending responsibilities of the layer. When the link layer receives a packet it will first verify the cyclic redundancy check (CRC) sum of the packet. If this check fails, the packet will be dropped. The next check performed is the destination MAC address check which determines if the packet is actually for this node. If the destination MAC address matches the network node, then the payload of the packet is handed off to the layer above it. Otherwise, the packet is simply dropped since it is not addressed for this node.\\

The \texttt{networkLayer(byte[], Transmit, Address)} method will packet a given payload in a \texttt{IPPacket} along with the source and destination IP address. It will determine the next hop to the destination through utility methods on the \texttt{Node} descriptor which will basically use the routing table for the node to determine the path. If no path can be found for the given destination, the default gateway address will be returned. After constructing the \texttt{IPPacket}, the packet object will be serialized (as byte array) and passed to the lower layer. When receiving a payload, this method will convert the byte array into an instance of the \texttt{IPPacket} class in order to extract and validate the header checksum. If the checksum fails, the packet will be dropped, otherwise, the payload of the packet is send to the layer above it.

\subsection{Network hosts}
The \texttt{transportLayer(byte[], Transmit, Address)} method segment a given payload when sending and reassemble multiple segments when receiving. When sending a given payload, it slices the payload into multiple segments based on the payload size. First it will check the length of the payload to make sure it needs to be sliced to MTU's size or not. If the payload size is greater than the MTU then it will slice the segments considering the header length. These segments has TCP header along with data. Source and destination ports are included in the header. SYN and FIN flags are used to identify segments starting and ending point. This payload will be sent to network layer for adding IP header and route this segment to right node. Sending parameters would be \texttt{networkLayer(byte[], Transmit, Address)} method. \\

When receiving a payload, this method will convert the payload into segments. It will then extract the chechsum to make sure segment is valid, if it's not valid it drops. If not, it keeps accumulating the segments into a buffer, along with sequence number. Then it will check for the FIN flag in the header to confirm that that's the last segment. If it's not FIN (it must be SYN), then it keeps assembling the segments to build a complete payload. Once the payload is built, it will be sent to application layer for creating a file in destination node.


\subsection{Network routers}
The functionality of the network layer is overridden for routers since they will just route packages to the next network or host. Similar to the inherited version of this method, it will start by verifying the IP header checksum. If the verification fails, the package is dropped. Otherwise, the correct network interface is found and the MAC address of that interface is used as the new source MAC address. The destination MAC address is then resolved through \texttt{Topology arpResolve(IP)},  and the payload is sent to the \texttt{linkLayer(byte[], Transmit, Address)} method of the router.

\subsection{Topology}
The topology for the network described by the assignment is fixed (as seen in figure \ref{fig:topo} of appendix \ref{appendix:topology}). In order to avoid having to hard-code information extracted from this topology (routing tables, MAC tables, etc), we decided to represent the topology in an external format as seen in the second part of appendix \ref{appendix:topology}. This will also allow us to change the topology of the network without having to change the implementation. Each node in the topology is represented by a corresponding node in the \textit{JSON} file including relevant information (see table below). Furthermore, the topology representation also contains the hardware address lookup table for the network.

\begin{center}
\begin{tabular}{l|{c}|{l}}
\textbf{Element} & \textbf{Type} & \textbf{Description} \\
\hline
id & Host, Router & Hostname for the network node \\ \hline
ip & Host & IP address for a \textit{host} \\ \hline
mask & Host & Network mask for a \textit{host} \\ \hline
mac & Host & Hardware address for a \textit{host} \\ \hline
mtu & Host & Maximum transmit unit for a \textit{host} \\ \hline
gateway & Host & Default gateway for a \textit{host} \\ \hline
routing & Host, Router & Routing table for the network node \\ \hline
ports & Router & Array of network interfaces for a \textit{router} \\ \hline
port $\to$ ip & Router & IP address for a given port on a \textit{router} \\ \hline
port $\to$ mask & Router & Network mask for a given port on a \textit{router} \\ \hline
port $\to$ mac & Router & Hardware address for a given port on a \textit{router} \\ \hline
port $\to$ mtu & Router & Maximum transmit unit for a given port on a \textit{router} \\ \hline
\end{tabular}
\end{center}

The topology representation is mapped to corresponding Java classes using a 3rd-party library called \texttt{Jackson}, which sole purpose is to process the \texttt{JSON} data format. The Java package \texttt{edu.fit.cs.computernetworks.topology} contains the mapped types which is briefly explained below.

\begin{description}
\item[Topology] is the root class of the topology model.
\item[Node] defines an abstract superclass for both Host and Router.
\item[Host] describes a network host.
\item[Router] describes a network router.
\item[Port] describes a network interface on a Router.
\item[RoutingEntry] describes an entry in a node's routing table.
\item[MACTableEntry] describes an entry in the global ARP table.
\end{description}

\section{Data flow}
Each node in the topology will be mapped to system threads and nodes of type \textit{host} will monitor a filesystem directory for files to send across the network. The name of the monitored directory is based on the node identifier such that for a given node, \textbf{A}, the corresponding directory would resolve to \textbf{/tmp/A} (where /tmp is the default root directory, which may be overridden by application arguments). \\

As soon as a file has been placed/discovered inside the monitored folder, the responsible thread reads the file into a byte array. It uses the filename to determine which host to send the file to, and resolves the IP address if this host through the \texttt{resolve(String)} utility method on the \texttt{Topology} class. The resulting destination IP along with its own IP address and the byte array payload will then be sent to the transport layer. \\

When receiving assembled data from the transport layer, the application layer will generate a generic file with the \textit{file-xx.bin} name pattern where the \textit{xx} part defines an arbitrary sequence number for files written. The assembled data will then simply be written to the new file to conclude the data flow.

\newpage
\section{Execution}
This project is managed by the \texttt{Gradle} build tool and depends on a couple of external libraries such as \texttt{Jackson}, \texttt{commons-lang}, \texttt{commons-io} for runtime dependencies, and \texttt{JUnit}/\texttt{Mockito} for test dependencies. The build tool will make sure that the proper versions of these libraries are downloaded and made available on the different classpaths. \\

The source distribution accompanying this report includes the build tool wrappers such that the build tool does not have to be installed on the target system. Invoking the build script wrappers will actually download the binaries needed to execute the \texttt{Gradle} build. Once the build is invoked, it will download the required dependencies mentioned above and place them in a local filesystem cache. This essentially means that the first build (including the download of the \texttt{Gradle} binary) will take a little bit of time, but subsequent builds will be fast. \\

The following steps explains how to compile and execute the application on the target system:

\subsection{Windows Environment:}
\begin{enumerate}
\item Open a terminal (command prompt) and navigate to the project's root folder
\item Execute the following command: \emph{gradlew.bat build}
\item Create sample folders to send files between nodes: \emph{Ex: {C:\textbackslash tmp\textbackslash A}, {C:\textbackslash tmp\textbackslash B}, {C:\textbackslash tmp\textbackslash C}}
\item Put files in the sample folders where the filename should equal the destination: \emph{Ex: {C:\textbackslash tmp\textbackslash A\textbackslash B.txt}}
\item Execute the following command to run the application: \emph{gradlew.bat run -ProotPath=C:\textbackslash tmp}
\end{enumerate}

\subsection{Unix Environment:}
\begin{enumerate}
\item Open a terminal and navigate to the project's root folder
\item Execute the following command: \emph{gradlew build}
\item Create sample folders to send files between nodes: \emph{Ex: /tmp/A, /tmp/B, /tmp/C}
\item Put files in the sample folders where the filename should equal the destination: \emph{Ex: /tmp/A/B.txt}
\item Execute the following command to run the application: \emph{gradlew run -ProotPath=/tmp}
\end{enumerate}

Notice the property \textit{‘rootPath’} that specifies the root path from where the threads will listen for files. The default path is set to \textit{‘temp/id’} where id is the name of the host node. See appendix \ref{appendix:runfirst} to \ref{appendix:runlast} for sample output from the application. The user should also be aware that the source distribution comes with a default file-set that can be used for simple testing. These files are located in the default \textit{temp} directory at the root of the project. Running \emph{gradlew run} without specifying the \textit{rootPath} property will use the file-set mentioned when starting the application. The file-set can be restored by running \emph{gradlew restore}.

\newpage
\appendix
\section{Assignment Topology} \label{appendix:topology}
\begin{figure}[H]
\centering
\includegraphics[scale=.65]{topology.png}
\caption{Network topology for the assignment}
\label{fig:topo}
\end{figure}
\bigskip
%\newpage
%\section{Topology Definition JSON} \label{appendix:topology}
\lstinputlisting[language=json,firstnumber=1]{../src/main/resources/topology2.json}
%\caption{Application topology defined in the JSON format}
%\label{fig:topoJson}
\centering\small{Application topology defined in the JSON format}
\newpage
\section{1 KB File transfer from Node-A to Node-B (B.txt)} \label{appendix:runfirst}
\lstinputlisting[language=json,firstnumber=1]{FromAtoB-1kb.log}
\newpage
\section{1 KB File transfer from Node-B to Node-C (C.txt)}
\lstinputlisting[language=json,firstnumber=1]{FromBtoC-1kb.log}
\newpage
\section{121 KB File transfer from Node-A to Node-C (C.jpg)}
\lstinputlisting[language=json,firstnumber=1]{FromAtoC-121kb.log}
\newpage
\section{276 KB File transfer from Node-B to Node-A (A.gif)}
\lstinputlisting[language=json,firstnumber=1]{FromBtoA-276kb.log}
\newpage
\section{4 KB File transfer from Node-C to Node-B (B.xls)}
\lstinputlisting[language=json,firstnumber=1]{FromCtoB-4kb.log}
\newpage
\section{13 KB File transfer from Node-C to Node-A (A.pdf)} \label{appendix:runlast}
\lstinputlisting[language=json,firstnumber=1]{FromCtoA-13kb.log}

\newpage
%\bibliographystyle{plain}
%\bibliography{critique2}
\begin{thebibliography}{30}
\bibitem{assignment} Dr. Marco Carvalho, \emph{CSE5231 - Class Project}, Florida Institute of Technology, Fall 2013.
    
\end{thebibliography}

\end{document}